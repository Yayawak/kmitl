\documentclass{article}
\usepackage{amsmath, graphicx, import, cancel}
% \usepackage{amsthm, xcolor, caption, subcaption}
% \usepackage{lineno}
% Loop
\usepackage{pgffor, multido}

% \linenumbers
% \author{Apisit Thaweboon \: 65050988}
\title { SEPERABLE DIFFERENTIAL }

\begin{document}
% \renewcommmand{\thesubsection}{\thesection.\alpha{subsect}}

\section*{46 ECOMONIC}


\subsection{Formulate a mathematical model in the form of an initial-value problem that rpresents the "flow" of the new currency into circulation}
\begin{table}
    \centering
    \begin{tabular}{ ||c|c|c|| }
        \hline
            & money & rateOfMoney \\
        \hline
        start state & \$10 billion & \$50 million \\
        any state & \$ 0 & ...
    \end{tabular}
\end{table}

flow = rate of in - rate of out \\

\section*{47* Tank of salt water}
A Tank contains 1000 L of brine with 15 kg of dissoved salt. Pure water enters the tank at rate of 10 L / min. The solution is kept thoroughly mixed and drains from the tank at the same rate \\
\textbf{a : How much salt is in the tank  after $t$ minutes ? } \\

\begin{split}
\begin{align}
    \text{rate salt in} &= $\frac{\text{number of salt}}{\text{time}}$ \\
    &= \frac{15kg}{1000L} \cdot
        \frac{10L}{min} \\
    &= \frac{15kg}{100min} \\
    &= 0.015 \frac{kg}{min}
\notag
\end{align}
\end{split}

\textbf{b : How much salt is in the tank  after 20 minutes ? } \\

% \multido{\i=3+4}{10}{\i\ }
% \import{./exers}{alcohol.tex}
% \import{./exers/}{alcohol.tex}
\import{exers/}{alcohol.tex}
% \import{/Users/rio/Desktop/kmitl/freshman/semester_2/calculus/w11_seperable_differential/lab/exers/}{alcohol.tex}

\end{document}
