% \LARGE 4.1.1
\section*{4.1.1}
% \raggedright 4.1.1
% \Large{4.1.1}
\textbf{
    Explain the difference between an absolute minimum \
    and a local minimum.
}
\par
   In mathematics, a minimum is a value of a function that is smaller than or equal to all other values of the function within a certain domain. There are two types of minimums: absolute minimums and local minimums.
\par
   An absolute minimum is the smallest value that a function can take over its entire domain. It is the global minimum of the function and is unique, meaning there is only one absolute minimum for a given function.
\par
   A local minimum, on the other hand, is a value of a function that is smaller than its nearby values but may not be the smallest value over the entire domain. A local minimum can occur at multiple points in the domain, and there may be other points in the domain where the function takes on smaller values.

